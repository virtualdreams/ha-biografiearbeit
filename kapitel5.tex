\section{Fazit}
\label{sec:k5_Fazit}

Aufgrund des demographischen Wandels und einer verbesserten medizinischen Versorgung hat sich die Altersstruktur der Bevölkerung in den letzten hundert Jahren in Deutschland stark verändert. Die vormalige Alterspyramide verjüngt sich nun eher nach unten. Die gestiegene Lebenserwartung führt zu einem vermehrten Auftreten chronischer Erkrankungen bis hin zur Pflegebedürftigkeit alter Menschen. Eine Krankheit, die in diesem Zusammenhang immer mehr in den Fokus der Öffentlichkeit gerät, ist die Demenz. Zurzeit sind ca. 1,3 Millionen Bürger in Deutschland daran erkrankt. Dieser organisch bedingte zunehmende Verlust kognitiver Fähigkeiten bedarf spezieller Behandlungsmöglichkeiten und Therapie. Biografiearbeit oder Erinnerungspflege hat sich in den vergangenen Jahren zunehmend bei der Therapie Betroffener etabliert. Biografiearbeit ermöglicht durch die Reflexion der Vergangenheit, die Gestaltung von Zukunft und die Einordnung der eigenen Person in ein bestehendes Kontinuum (vgl. Kapitel \ref{sec:k3_BiografiearbeitInDerAltenarbeit}). Die tiergestützte Biografiearbeit setzt an diesen Zielstellungen an und erweitert diese um aktivitätsfördernde tiergestützte Methoden.

Insbesondere Demenzkranke weisen häufig Kennzeichen von Identitätsverlust und sozialen Rückzug auf. Biografiearbeit vermag hier das "`Personensein"' durch die Befriedigung der Bedürfnisse von Beschäftigung, Identität und Einbeziehung wieder in den Vordergrund zu holen. Die tiergestützte Biografiearbeit ist hier eine Möglichkeit der aktivitätsorientierten Biografiearbeit, welche sich an den Erfahrungen und Ressourcen der Betroffenen orientiert. Neben dem Einsatz der variablen Bausteine ist insbesondere die positive Wirkung der Tiere auf die Psyche Betroffener, welche die tiergestützte Biografiearbeit so interessant für den therapeutischen Einsatz macht. Im Umgang mit Tieren wird gelacht, geschmust, seelische Sorgen werden abgemildert und man erfährt Momente der Entspannung und seelischer Ruhe. Diese Effekte werden als sehr positiv für Demenzkranke bewertet und sind zu fördern. 

Zusammenfassend kann man festhalten, dass die verschiedenen Aktivitäten mit Tieren im Rahmen der tiergestützten Biografiearbeit wichtige Ressourcen für die Demenzkranken darstellen, da sie sich förderlich auf Wohlbefinden und Lebenssituation der Betroffenen auswirken können. Neben der biografischen Tätigkeit bedarf es jedoch auch einem fundierten Wissen um das Krankheitsbild. Nur so kann ein geeigneter Zugang bei der Therapie und Erinnerungspflege dementer Personen geschaffen werden.

Um eine grundsätzliche Bewertung dieser Arbeit vornehmen zu können, bedarf es jedoch weiterer Forschung und einer diffizilen Auseinandersetzung mit Begleitfaktoren wie Organisation, Finanzierung oder Überlegungen eines berufs- oder krankheitsübergreifenden Einsatzes. Weiterhin wurden im Rahmen dieser Arbeit detaillierte Faktoren wie die Auswahl der Tiere und die Gruppenzusammenstellung der Betroffenen vernachlässigt. Eine vollständige Darlegung war in Form dieser Seminararbeit nicht möglich. 

Diese Form der Erinnerungspflege offeriert einem vielfältige Anregungen, welche man privat aber auch im späteren beruflichen Kontext anwenden könnte.

Abschließen möchte ich mit einer Aussage von Hegedusch und Hegedusch. Diese bekräftigt eindrücklich die Bedeutung tiergestützter Biografiearbeit bei der Erinnerungspflege demenzkranker Personen.

\begin{quotation}
Tiere sprechen Fähigkeiten beim Betroffenen an, die wenig oder gar nicht beeinträchtigt sind. Über die verschiedenen Sinne werden Effekte ausgelöst, die positiv auf den emotionalen, sozialen und kognitiven Status dieser Menschen wirken\footcite[72]{Hegedusch2007}.
\end{quotation}