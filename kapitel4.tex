\section{Tiergestützte Biografiearbeit}
\label{sec:k4_TiergestützteBiografiearbeit}

\subsection{Hinführung}
\label{sec:k4.1_Hinführung}

Das Konzept der tiergestützten Biografiearbeit wurde u. a. von Mandy Giruc beschrieben und richtet sich vorrangig an dementiell erkrankte Personen die institutionell betreut werden.

Tiergestützte Biografiearbeit ist ein gezielter Ansatz der Biografiearbeit, mit dessen Hilfe Demenzkranke in ihre Vergangenheit blicken und ihr Erfahrungswissen mit Tieren reaktivieren können\footcite[vgl.][32f]{Giruc2011}. Es handelt sich dabei um eine nichtmedikamentöse Therapiemöglichkeit, welches einen zielgerichteten und aktiven Gestaltungsprozess im Zusammenhang mit Tieren ermöglicht. Der Umgang mit den Tieren soll sich positiv auf das Selbstwertgefühl und die Stimmung auswirken, das Identitätsgefühl steigern, von Problemen ablenken aber auch die kognitiven Ressourcen aktivieren und fördern. Der ressourcenorientierte Ansatz richtet sich nach den Möglichkeiten, Erfahrungen und Potentiale der Teilnehmer und agiert nur mit vorher trainierten, toleranten und vertrauensvollen Tieren. 

Dabei ist es für mich weniger von Interesse warum welches Tier gewählt wurde und welche Rahmenbedingungen für einen möglichen Kontakt geschaffen werden müssen, sondern vielmehr die inhaltliche Ausgestaltung und die biografischen Möglichkeiten welche die Arbeit mit einem Tier gestattet. Nach einem kurzen Exkurs in das Themengebiet der Heim- und Haustiere soll im Folgenden das Konzept der tiergestützten Biografiearbeit, anhand ausgewählter Tierarten, praxisnah dargelegt werden. 

\subsection{Heim- und Haustiere}
\label{sec:k4.2_HeimUndHaustiere}

Bereits seit Urzeiten leben Menschen mit Tieren in enger Gemeinschaft und nutzen so deren Fähigkeiten und Fertigkeiten. Sei es nun der klassische Wachhund, die verschmuste Katze, der Wellensittich oder andere weit verbreitete Haustiere -- so ziemlich jeder Mensch hat eine mehr oder minder ausgeprägte Beziehung zu Tieren. 

Dabei soll jedoch zwischen Heim- und Haustieren unterschieden werden. Unter Heimtieren versteht man Tiere wie Katzen, Hunde und Vögel, die ausschließlich zum Vergnügen und Zuhause gehalten werden. Unter dem Begriff Haustiere werden auch sogenannte Nutztiere gefasst wie Ziegen, Hühner oder Schweine. Diese werden zwar vorrangig aus wirtschaftlichen Zwecken gehalten, können aber auch eine Beziehung zu einer Person entwickeln\footcite[vgl.][47f]{Leder2006} . Aufgrund des hohen menschlichen Bezuges sind sowohl Heim- als auch Haustiere grundsätzlich für den therapeutischen Einsatz denkbar.

\subsection{Exemplarische Darstellung der Tiergestützten Biografiearbeit}
\label{sec:k4.3_ExemplarischeDarstellungDerTiergestütztenBiografiearbeit}

\subsubsection{Tiergestützte Arbeit mit Hühnern}
\label{sec:k4.3.1_TiergestützteArbeitMitHühnern}

\paragraph{Hinführung}
\label{sec:k4.3.1.1_Hinfuehrung}

Obwohl Federtiere keine klassischen Streicheltiere sind, so haben ältere Menschen oft eine intensive Beziehung zu diesen Tieren als jüngere Personen. Viele von ihnen sind auf dem Land aufgewachsen, wo der Umgang mit diesen Nutztieren normal und alltäglich war, zudem haben Hühner zu Kriegszeiten zur Ernährung der Familie beigetragen. 
Hühner sind sehr aktive und kontaktfreudige Tiere. Zudem bieten sie demenzkranken Personen viele interessante Reize, vom Sehen über das Fühlen bis zum Hören\footcite[vgl.][33f]{Giruc2011}. Für den Einsatz in der tiergestützten Biografiearbeit sollte man vorrangig auf Zwergseidenhühner oder Zwergcochins zurückgreifen, da diese besonders ruhig und ausgeglichen sind und so eine relativ problemlose Arbeit mit den Demenzkranken ermöglicht wird.

\paragraph{Eine mögliche Gruppenstunde}
\label{sec:k4.3.1.2_EineMöglicheGruppenstunde}

Vor einer geplanten Gruppenstunde unter Einsatz von Hühnern bedarf es einer konkreten Vorbereitung und Planung. Dazu bedarf es der Bereitstellung passender Materialien zum Thema Huhn in Form von Bildmaterialien, gekochten Eiern, Federn, Körner oder auch bestimmte Pflanzen. Zudem benötigt man eine abschließbare Räumlichkeit und ein tiergerechtes Gehege\footcite[vgl.][61-67]{Giruc2011}. 

Die eigentliche Stunde wird in einen Einstieg, Hauptteil und Abschluss untergliedert. Während des Einstieges erfolgt die Begrüßung, man kündigt das Thema an und stellt die Tiere vor. Im Hauptteil greift man jetzt auf die persönlichen Erfahrungen der Teilnehmer zurück und bindet diese aktiv ins Geschehen ein. Sei es nun die Erinnerung an die eigene Hühnerhaltung, die Sinneswahrnehmung von Eiern, Pflanzen- und Körnerbestimmungen, das Schmecken von Eiergerichten oder die Begegnung mit dem Tier -- vorhandene Ressourcen werden so gefördert und mögliche Anspannungen reduziert. Die Teilnehmer können sich unterstützt durch Abbildungen an den Werdegang vom Ei zum Huhn annähern oder passende Bauernweisheiten erinnern und besprechen. Ein Beispiel hierfür wäre "`Kräht der Hahn auf dem Mist, ändert sich das Wetter oder es bleib wie es ist."' 

Dauer und Intensität der Themen richten sich nach den Interessen der Teilnehmer. Ziel ist es, die Beteiligten aktiv einzubeziehen und zu beschäftigen um so einen besseren Identitätsbezug zu ermöglichen. Zum Schluss werden die erlebten Eindrücke reflektiert, die Tiere werden verabschiedet und auf das kommende Thema eingestimmt.
\newpage

\subsubsection{Tiergestützte Arbeit mit Hunden}
\label{sec:k4.3.2_TiergestützteArbeitMitHunden}

\paragraph{Hinführung}
\label{sec:k4.3.2.1_Hinfuehrung}

Hunde gehören schon seit jeher zu den wichtigsten tierischen Begleitern des Menschen. Aufgrund seiner Anhänglichkeit, Treue und Loyalität gegenüber dem Menschen, scheinen Mensch und Hund einander sehr verbunden\footcite[vgl.][92]{Ochsenbein2005}. Kurzum, der Hund gehört zu den Tierarten mit denen sich der Mensch am leichtesten und eingehendsten verständigen kann\footcite[vgl.][95]{Ochsenbein2005}. Für die Arbeit mit Demenzkranken muss der ausgewählte Hund zudem Merkmale wie Zuverlässigkeit, Menschenbezogenheit, Gehorsam und Gelassenheit besitzen\footcite[vgl.][42]{Giruc2011}.

\paragraph{Eine mögliche Gruppenstunde}
\label{sec:k4.3.2.2_EineMöglicheGruppenstunde}

Wie auch bei den Hühnern gehört eine eingehende Planung und Organisation der einsetzbaren Materialien vor die Durchführung einer hundgestützten Stunde. Da sich auch Einstieg und Reflexion inhaltlich gleichen, hier noch ein paar Hinweise zur Ausgestaltung des Hauptteils. Auch hier sind wieder Materialien denkbar, die das Thema Hund widerspiegeln. Zum einen werden Decken, Leckerlis und ein Wassernapf für den Hund benötigt, zum Anderen können verschiedene Hundeassoziationen zum Einsatz kommen. Darunter zählen Bürsten und Kämme, Zutaten für einen "`Kalten Hund"', Hundespielzeug oder auch eine "`Lassie"'-DVD\footcite[vgl.][106-111]{Giruc2011}. Im Laufe der Stunde kann man erneut an die eigene Hundehaltung erinnern, die Tiere streicheln oder in Kontakt mit ihnen treten, Hunderassen zuordnen lassen oder auch passende Sprichwörter wie "`auf den Hund gekommen"' einbeziehen. Insbesondere für frühere Hundehalter sind diese Stunden sehr wertvoll.

\subsubsection{Tiergestützte Arbeit mit Kaninchen}
\label{sec:k4.3.3_TiergestützteArbeitMitKaninchen}

\paragraph{Hinführung}
\label{sec:k4.3.3.1_Hinfuehrung}

Kaninchen sind aufgrund ihres weichen Fells, den rundlichen Augen und den langen Ohren schon lange als geeignete Besuchstiere bekannt. Zudem sind sie besonders kontakt- und berührungsfreudige Tiere, was sich im Kontakt mit Demenzkranken als ein Vorteil für beide Seiten herausstellen kann\footcite[vgl.][40]{Giruc2011}. Fühlen sich die Kaninchen in Obhut der Beteiligten wohl, so kann man eingehende Kuschelsequenzen erleben, die beim Betroffenen neben den Zielstellungen der tiergestützten Biografiearbeit (Einbeziehung, Beschäftigung und Identität) auch weitere positive Effekte auslöst, wie das Abmildern psychischer Anspannungen oder das Gefühl von Wohlbefinden\footcite[vgl.][74]{Leder2006}.

\paragraph{Eine mögliche Gruppenstunde}
\label{sec:k4.3.3.2_EineMöglicheGruppenstunde}

Zur Vorbereitung der Stunde kann man beim Thema Kaninchen entsprechende Bilder, Stofftiere, Grünfutter, Tastsäckchen mit Inhalt (Heu, Möhre, Äpfel etc.), weiche Bürsten oder "`tierische"' Geschichten verwenden\footcite[vgl.][90-95]{Giruc2011}. Diese können im Hauptteil dafür genutzt werden, um an die eigene (mögliche) Kaninchenhaltung zu erinnern und um in einen intensiven Kontakt mit dem Tier zu treten. Hier wären Fragestellungen denkbar wie "`Können Sie sich noch an die langen Ohren der Tiere erinnern?"' oder "`Wie haben Sie die Tiere gehalten?"'. Über diese Form der kognitiven Aktivierung können Erinnerungen wachgerufen werden, Emotionen geweckt oder auch Aufmerksamkeit erregt werden. Weiterhin können feinmotorische Fähigkeiten mit Hilfe der Tastsäckchen trainiert werden. Es obliegt der Kreativität der umsetzenden Person, aber auch dem Vorwissen und der Interessen der Beteiligten, ein adäquates Angebot zu gestalten.

\subsection{Begleitende Methoden (variable Bausteine)}
\label{sec:k4.4_BegleitendeMethodenVariableBausteine}

Wie bereits angesprochen haben Demenzkranke ein besonderes Bedürfnis nach sinnvoller Beschäftigung. Neben dem Einsatz der Tiere sind daher die sogenannten variablen Bausteine wesentlicher Bestandteil der tiergestützten Biografiearbeit. Darunter werden Beschäftigungen zusammengefasst wie Lieder singen, Texte hören, Bilder ansehen, Assoziationen bilden, Sprichwörter formulieren, Bewegungsaktivitäten oder wahrnehmungsfördernde Übungen\footcite[vgl.][50f]{Giruc2011}. All diese Aufgaben dienen dazu das Selbstwertgefühl zu steigern, emotionales Wohlbefinden auszulösen, aber auch um verbleibende Selbstständigkeit zu erhalten. Zudem werden so die vorhandenen Ressourcen der Betroffenen genutzt, was sich wiederum an den therapeutischen Zielstellungen bzw. Bedürfnissen von Demenzkranken orientiert. Diese (Einbeziehung, Beschäftigung und Identität) wurden bereits unter Punkt \ref{sec:k2.2.3_Beduerfnisse} dargelegt.
In Situationen wo den Teilnehmern keine Tiere zur Verfügung stehen, kann diese Form der Erinnerungspflege durchaus auch nur mit diesen begleitenden Methoden zustande kommen.

\subsection{Die positive Wirkung von Tieren}
\label{sec:k4.5_DiePositiveWirkungVonTieren}

Nachdem die Intention und Umsetzung der tiergestützten Biografiearbeit nun in Auszügen dargelegt worden ist, möchte ich abschließend auf den psychologischen Effekt von Tieren eingehen. Dieser Aspekt beeinflusst maßgeblich den Erfolg von Biografiearbeit und rückt das Tier in den Fokus der Therapie.

Otterstedt, Leder und weitere Autoren verweisen in diesem Zusammenhang auf die positive Beeinflussung von Tieren auf die Psyche alter Menschen. Durch die Befriedigung zentraler seelischer Bedürfnisse können Tiere zu einer Steigerung der Lebensqualität beitragen. Die Betroffenen empfinden häufig Trost und Ablenkung, die seelischen Sorgen werden abgemildert, Beruhigung- und Entspannungsphasen setzen ein und die Menschen lachen häufiger im Kontakt mit dem Tier\footcite[vgl.][27f]{Otterstedt2001}\footcite[vgl.][75]{Leder2006}. Diese Zufriedenheit beruht auf physischen und chemischen Reaktionen des menschlichen Organismus und verursacht häufig ein vermindertes Stress- oder Schmerzempfinden, aber auch freudige bis euphorische Gefühle\footcite[vgl.][75]{Leder2006}. 

Im Kontakt mit den Tieren werden außerdem kognitive Leistungen gefördert, da diese nach Reaktion und Aufmerksamkeit verlangen. Diese Form der Konzentrationsförderung kommt insbesondere kranken Menschen (mit Demenz) zugute und bekräftigt den Ansatz der tiergestützten Biografiearbeit im Kontext der Erinnerungspflege mit Demenzkranken\footcite[vgl.][76]{Leder2006}.
