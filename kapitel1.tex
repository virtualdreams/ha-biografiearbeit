\section{Einleitung}
\label{sec:k1_Einleitung}

\begin{quotation}
"`Eine Identität zu haben, bedeutet zu wissen, wer man ist, im Erkennen und im Fühlen. Es bedeutet, ein Gefühl der Kontinuität mit der Vergangenheit und demnach eine Geschichte, etwas, das man anderen präsentieren kann, zu haben."'
\end{quotation}

\begin{flushright}
(Kitwood 2008, S.125 In: Giruc 2011, S.19)
\end{flushright}

Die Begegnung mit alten und dementen Menschen löst die verschiedensten Reaktionen bei uns aus. Sei es nun Mitgefühl, Unsicherheit oder Ablehnung -- viele Personen haben im Rahmen ihrer beruflichen Arbeit, im Freundes- oder Familienkreis oder einfach im Vorbeigehen eigene Erfahrungen im Erleben von Demenzkranken gesammelt. Diese Personen sind zwar reich an Erinnerungen und Erfahrungen, jedoch ist der Zugang zu diesen Informationen oft erschwert.
 
Die Intention zur Bearbeitung dieses Themas liegt in meiner eigenen Geschichte begründet. Ich entstamme einer vormaligen Großbauernfamilie, welche früher Landwirtschaft betrieb aber auch viele Haus- und Hoftiere besaß. Mein Uropa lebte bis zu seinem Tod auf dem ehemaligen Familiengehöft, mittlerweile ohne Tiere. In den letzten Jahren seines Lebens wurde es zunehmend schwer mit ihm zu interagieren da er an fortschreitender Demenz litt. Kam man auf altbekannte Themen zu sprechen, begann er jedoch zu Erzählen. Als er starb, war die Tür zu diesen Erinnerungen unwiderruflich zugefallen. Viele Geschichten blieben nun verborgen. Im Nachhinein suchte ich nach Strategien um adäquater mit solchen Personen in Kontakt treten zu können. Dabei stieß ich u. a. auf die tiergestützte Biografiearbeit. Sie ermöglicht eine besondere Form des Zugangs zu Demenzkranken und soll Inhalt meiner Arbeit sein.

Um dieses Thema in seiner Komplexität handhabbar zu machen, habe ich mich dafür entschieden es thematisch aufzugliedern. Zunächst werden die Begriffe von Alter und Demenz in seinen Grundzügen dargelegt, um so zu einer wesentlichen Vorstellung über das genannte Krankheitsbild zu gelangen. Um anschließend das Thema der Tiergestützten Biografiearbeit darstellen zu können, wird vorab Biografiearbeit selbst kurz erläutert und darauf aufbauend die tiergestützte Form dargelegt. Da diese Form der Therapie sehr vielfältig gestaltet werden kann, kann nur eine exemplarische Darstellung mit ausgewählten Tierarten vorgenommen werden. Bei der Auswahl der Tierarten habe ich mich u. a. an den Erinnerungen meines Uropas orientiert. Dennoch sind viele weitere Tierarten denkbar. Zum Schluss werden ein paar begleitende Methoden zur tiergestützten Biografiearbeit aufgezeigt.
Ziel meiner Arbeit ist die grundlegende Darstellung der tiergestützten Biografiearbeit mit Demenzkranken als eine Form des Zugangs und der Erinnerungspflege. Aufgrund des begrenzten Rahmens und der exemplarischen Veranschaulichung ist eine vollständige Darlegung dieser Therapieform und der genannten Begleitfaktoren jedoch nicht möglich. Daher werden nur ausgewählte Aspekte zur Beobachtung herangezogen und thematisch unrelevante Aspekte vernachlässigt. Der Anspruch zur thematischen Vollständigkeit besteht somit nicht.

Zur Bearbeitung des Themas, habe ich primär die umfangreiche Literatur der erziehungswissenschaftlichen Universitätsbibliothek genutzt. Dabei habe ich mich inhaltlich stark am Werk von Mandy Giruc mit dem Titel "`Tiere, mit denen wir lebten"' orientiert, da  das Thema der tiergestützten Biografiearbeit hier besonders anschaulich und praxisorientiert vorgestellt wird. 
