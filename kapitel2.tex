\section{Klärung der Begleitfaktoren}
\label{sec:k2_Klaerung der Begleitfaktoren}

\subsection{Alter}
\label{sec:k2.1_Alter}

Auch wenn einem die Nutzung des Begriffs Alter vermeintlich normal und vertraut erscheint, so ist er dennoch schwer zu definieren. Früher wurde die Lebensphase Alter in erster Linie vom Eintritt in das politische Rentenalter vorreguliert. Doch aufgrund politischer und gesellschaftlicher Veränderungen und dem sich zunehmend nach hinten verlagerten Renteneintrittsalter, verliert dieser Punkt zunehmend an Bedeutung. Alter ist hingegen heute zu einem vielschichtigen und schwer zu bestimmenden Begriff geworden. Es gibt verschiedenste Versuche von Buchautoren und Internetquellen diesen Ausdruck an die moderne Welt anzupassen und ihn in seiner Komplexität handhabbar zu machen. Eine mögliche Altersdefinition wird auf der Internetseite von Wikipedia angeboten. Die dortige Formulierung lautet wie folgt: "`Unter dem Alter versteht man den Lebensabschnitt rund um die mittlere Lebenserwartung des Menschen, also das Lebensalter zwischen dem mittleren Erwachsenenalter [Anmerkung: zwischen 35-65Jahre] und dem Tod. Das Altern in diesem Lebensabschnitt ist meist mit einem Nachlassen der Aktivität und einem allgemeinen körperlichen Niedergang verbunden."'\footcite{Wikipedi2014} 

Wie man diesem Textabschnitt entnehmen kann, orientiert sich der Verfasser hier primär am biologischen Alter der Person. Diese Definition erscheint mir unzureichend, soll aber aufgrund der gesellschaftlichen Popularität der Seite erwähnt werden. Im Werk von Backes und Clemens "`Lebensphase Alter"', wird die selbige hingegen mehr als soziales und gesellschaftliches Konstrukt verstanden. Das bedeutet, dass in unserer heutigen Leistungsgesellschaft der Begriff Alter "`als Phase der eingeschränkten Leistungsfähigkeit, des nicht mehr Mithaltenkönnens, des Ausrangiertwerdens, aber auch als Phase des Zurückblickens auf die erbrachte Leistung [\punkte]"'\footcite[13]{Backes1998} begriffen wird. Wann genau diese Lebensphase für eine Person beginnt, ist stets individuell zu betrachten, ist sie doch abhängig von den begleitenden körperlichen, psychischen, sozialen und gesellschaftlichen Gegebenheiten und ist somit nicht klar abgrenzbar\footcite[15]{Backes1998}. 

Eine Erkrankung die häufig mit dem Alter einhergeht, ist die Demenz. Diese soll im nächsten Punkt kurz charakterisiert werden.

\subsection{Demenz}
\label{sec:k2.2_Demenz}

\subsubsection{Hinführung}
\label{sec:k2.2.1_Hinfuehrung}

Das allgemeine Erkrankungsrisiko eines Menschen nimmt mit steigendem Alter stetig zu. Eine aufgrund des demografischen Wandels immer mehr an Bedeutung gewinnende Erkrankung ist in diesem Zusammenhang die Demenz. In Deutschland sind derzeit ca. 1,3 Millionen Menschen daran erkrankt, wobei nur ca. 3\% davon unter 65 Jahre alt sind \footcite[9]{Schwarz2009}. Da sich das Risiko für eine dementielle Erkrankung ab 65 Jahren alle 5 Jahre nahezu verdoppelt, ist sie primär den altersbedingten Erkrankungen zuzuordnen. Dennoch können auch jüngere Personen daran erkranken. 

Ist eine Demenz eingetreten, spricht man von einem "`organisch bedingten, meist über Jahre fortschreitenden Verlust von intellektuellen Fähigkeiten mit darauf basierenden Beeinträchtigungen im Alltag"'\footcite[11]{o.A.2013}. Um mit diesem Erkrankungsbild adäquat umgehen zu können, bedarf es genauer Kenntnisse der Symptome, Stadien und Bedürfnisse erkrankter Personen. Nur so kann eine patientengerechte Betreuung gewährleistet werden aber auch die später thematisierte tiergestützte Biografiearbeit gelingen. In dieser Arbeit wird das Thema der Demenz aus inhaltlichen Gründen jedoch nur in seinen Grundzügen dargelegt. Unterpunkte wie Ursachen oder medikamentöse Behandlung der Erkrankung werden aufgrund der fehlenden Relevanz vernachlässigt.

\subsubsection{Stadienabhängige Symptome}
\label{sec:k2.2.2_StadienabhängigeSymptome}

Die Demenz bezeichnet keine einzelne Krankheit sondern beinhaltet eine Vielzahl verschiedener Symptome, die in ihrer Ausprägung das Erkrankungsbild formen. Die Krankheitszeichen variieren dabei abhängig vom Stadium der Erkrankung. Um ein grundlegendes Verständnis von Demenz zu schaffen, werden die einzelnen Stadien in ihren Grundzügen nun kurz dargelegt.

\begin{itemize}
\item Leichte Demenz: Die krankhaften Prozesse der Demenz haben zwar bereits ein paar Jahre zuvor begonnen, doch zeigen sich jetzt die ersten Symptome in Form von scheinbar Unwichtigkeiten. Die Personen vergessen Kleinigkeiten; sie haben Schwierigkeiten sich Namen, Orte oder eben erfolgte Handlungen zu merken und haben zunehmend Wortfindungsschwierigkeiten\footcite[15]{Giruc2011}. Zudem lässt die zeitliche und räumliche Orientierung nach und viele Erkrankte leiden unter Gefühlsschwankungen, Antriebslosigkeit oder Ängstlichkeit. Die Betroffenen verspüren nun ihre eigenen Defizite, empfinden häufig Hilflosigkeit und distanzieren sich zunehmend von ihrer Umwelt\footcite[15f]{Giruc2011}.

\item Mittelschwere Demenz: Die Symptomausprägung der Krankheit nimmt nun deutlich zu, so das alltägliche Handlungen wie das Anziehen, das Essen oder auch der Toilettengang nur mit Hilfestellung erfolgen können. Das Raum- und Zeitgefühl schwindet zunehmend und Erkrankte verirren sich auch im eigenen Haus. Die sprachlichen und körperlichen Einbußen sind nun klar erkennbar. "`Undefiniertes rhythmisches Gemurmel"' oder ein unsicheres Gangbild kennzeichnen jetzt die Erkrankten\footcite[16f]{Giruc2011}. Der Betroffene vergisst nun auch seine Erkrankung und etwaige Anspannungen und Unsicherheiten lösen sich auf.

\item Schwere Demenz: Der Erkrankte hat nun keine Kontrolle mehr über seinen eignen Körper. Der körperliche und geistige Niedergang  hat nun den höchsten Ausprägungsgrad erreicht. Gegenstände und Personen können nun nicht mehr zugeordnet werden, die Bewegungsabläufe kommen zum Erliegen und enden häufig in der Bettlägerigkeit, aber auch die Körperfunktionen wie die Kontrolle von Blase und Darm oder das Schlucken fallen zunehmend schwer\footcite[17]{Giruc2011}. Eine professionelle Pflege ist nun meist erforderlich.
\end{itemize}

\subsubsection{Bedürfnisse }
\label{sec:k2.2.3_Beduerfnisse}

Wie man den eben genannten Informationen entnehmen kann, ist die Demenz nicht nur eine organische Veränderung des Gehirns, sondern geht oft einher mit psychischen Veränderungen wie Unsicherheit, Hilflosigkeit und Angst, was zu einer Abkehr von der Umwelt und zunehmender Isolation führen kann. Depressive Verstimmungen oder Depressionen können die Folge sein. 
 
Der personenzentrierte Ansatz von Tom Kitwood ermöglicht in diesem Zusammenhang eine andere Betrachtungsweise für den angemessenen Umgang mit dementiell erkrankten Menschen und rückt das "`Personensein"' der Betroffenen wieder in den Vordergrund\footcite[17f]{Giruc2011}\footcite[67f]{o.A.2013}. Die fehlende Autonomie und Rationalität der Erkrankten führt Kitwood zufolge zu einer Depersonalisierung. Diese Art von  Abwehrmechanismus ist in der eigenen Angst vor Gebrechlichkeit und geistigem Abbau und der damit verbundenen Abhängigkeit begründet. Um dem entgegen zu wirken und das "`Personensein"' wieder in den Vordergrund zu stellen, verweist er auf die Bedeutung der grundlegenden Bedürfnisse von Trost, primärer Bindung, Einbeziehung, Beschäftigung und Identität\footcite[18f]{Giruc2011}. Die Bedürfnisse von Einbeziehung, Beschäftigung und Identität werden nun exemplarisch näher erläutert, da sie im späteren Verlauf der Arbeit, im Kontext der tiergestützten Biografiearbeit, erneut aufgegriffen werden.

\begin{itemize}
\item Einbeziehung: Der Mensch ist grundsätzlich für ein Leben in der Gruppe veranlagt. Nur so konnte früher das Überleben gesichert werden. Kann die soziale Interaktivität aufgrund der Erkrankung nicht mehr gewährleistet werden, kommt es häufig zum persönlichen Rückzug. Erhält der Mensch hier jedoch Befriedigung kann er erneut zum aktiven und eigenständigen Teil der Gruppe befähigt werden.
\item Beschäftigung: Etwas Sinnvolles zu tun, egal was, vermeidet Langeweile und Apathie. Eine aktive Teilnahme und Hilfsbereitschaft offenbaren hier das Bedürfnis.
\item Identität: "`Eine Identität zu haben, bedeutet zu wissen, wer man ist, im Erkennen und im Fühlen. Es bedeutet, ein Gefühl der Kontinuität mit der Vergangenheit und demnach eine Geschichte, etwas, das man anderen präsentieren kann, zu haben."' (Kitwood 2008, S.125 In: Giruc 2011, S.19) Dieser Satz verdeutlicht die Bedeutung der eigenen Lebensgeschichte im Kontext der eigenen Identität, aber auch im Rahmen der institutionellen Versorgung und Unterbringung.
\end{itemize}

\subsubsection{Therapie}
\label{sec:k2.2.4_Therapie}

Aufgrund der fehlenden Relevanz zum Thema wird die medikamentöse Behandlung von Demenz hier vernachlässigt und nur auf die nicht-medikamentöse Therapie eingegangen.

Grundlegende Zielsetzung der Therapie ist die Verbesserung der kognitiven Fähigkeiten, der Erhalt der selbstständigen Lebensführung aber auch die Förderung der Lebenszufriedenheit\footcite{Hegedusch2007}. Neben einem psychomotorischen Training können auch kommunikations- oder sinnesanregende Übungen zu dieser Zielstellung beitragen. 

Der Einsatz der Biografiearbeit soll hier auch genannt werden, wissend dass die Therapiezuordnung sehr umstritten ist. Sie zielt jedoch auf den Erhalt von Identitätsgefühl, hat so eine hohe Bedeutung bei der Arbeit mit Demenzkranken und soll nun anhand der tiergestützten Biografiearbeit näher erläutert werden.
