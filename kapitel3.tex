\section{Biografiearbeit in der Altenarbeit}
\label{sec:k3_BiografiearbeitInDerAltenarbeit}

Um das Thema der tiergestützten Biografiearbeit besser verorten zu können, soll zunächst Biografiearbeit an sich kurz definiert werden. 

Trotz unzähliger Literatur zum Thema ist es gar nicht so leicht eine schlüssige Definition von Biografiearbeit zu finden. Nach eingehender Recherche habe ich mich für die Definition von Ruhe entschieden, da sie die Kernelemente dieser Arbeit klar formuliert.

\begin{quotation}
"`Biografiearbeit ist der Versuch, Mensch--Sein als Körper, Geist und Seele in den individuellen, gesellschaftlichen und tiefenpsychologischen Dimensionen wahrzunehmen. In der Rückschau auf das eigene Leben, wächst Verständnis für das Eigene. Biografiearbeit ermöglicht, sich sinnhaft als Bestandteil eines Kontinuums zu definieren."'
\end{quotation}

\begin{flushright}
(Ruhe 1998, S. 134 In: Miethe 2011, S. 22)
\end{flushright}

Diese Formulierung verdeutlicht den Anspruch von Biografiearbeit den Menschen in seiner Ganzheitlichkeit zu betrachten und die Reflexion der Vergangenheit zur Gestaltung der Zukunft zu nutzen.
Im Rahmen der Altenarbeit oder in der Arbeit mit Demenzkranken wird auch häufig von Erinnerungspflege gesprochen. Bei Personen, deren Gedächtnisleistung zunehmend nachlässt, welche auf Hilfe angewiesen sind und zudem aus ihrer vertrauten Umgebung genommen wurden, ist die Gefahr des Identitätsverlustes und des sozialen Rückzuges sehr hoch. Der Zugriff auf das eigene Gedächtnis kann hier den Schlüssel zu vorhandenen Kompetenzen bilden und zudem Sicherheit und Selbstvertrauen vermitteln\footcite[vgl.][25]{Giruc2011}. Biografiearbeit knüpft an den Ressourcen und Bedürfnissen der Menschen an, hilft Einsamkeit zu überwinden und dem Leben wieder einen Sinn zu geben. Gefühle von Glück, Stolz, Freude und Lebensmut können so wieder erlebt oder reaktiviert werden\footcite[vgl.][75]{Schwarz2009}. Zielstellungen die, wie bereits erwähnt, auch bei der Arbeit mit Demenzkranken von hoher Bedeutung sind. 

Wie diese Zielstellungen erreicht werden können, soll anhand der tiergestützten Biografiearbeit nun exemplarisch dargelegt werden.