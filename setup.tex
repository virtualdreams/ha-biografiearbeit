%%% Dokumentenklasse Artikel
%%% 12pt
%%% A4
%%% Titelseite
%%% Nummerierung ohne endende Punkte zB. 1. ; 1.1.
%%% einseitig
\documentclass[fontsize=12pt,paper=a4,titlepage,numbers=noenddot,twoside=false,bibliography=totoc,toc=listof]{scrartcl}

%%% Schriftart, Kodierung und Sprache
\usepackage[T1]{fontenc}
\usepackage[utf8]{inputenc}
\usepackage[ngerman]{babel}

%%% Schriftart
%\usepackage{ae}
\usepackage{times}

%%% Space :)
\usepackage{setspace}

%%% Multirow in Tabellen (miktex only?)
\usepackage{multirow}

%%% Grapfikpaket
\usepackage{graphicx}

%%% Randeinstellungen
\usepackage[left=30mm,right=30mm,bottom=25mm,top=25mm]{geometry}

%%% Blindtext
\usepackage{lipsum}

%%% Footer und Header
%% headsepline - Headerlinie
%% footsepline - Fusslinie
%% mehr unter http://get-software.net/macros/latex/contrib/koma-script/doc/scrguide.pdf
\usepackage[automark]{scrlayer-scrpage}

%%% Formelkram falls gebraucht
%\usepackage{amsmath}
%\usepackage{amsfonts}
%\usepackage{amssymb}

%%% Listen
\usepackage{enumerate}
\usepackage{listings}

%%% Listeneinträge sollen alle mit Bullets anfangen
\renewcommand{\labelitemi}{$\bullet$}
\renewcommand{\labelitemii}{$\bullet$}
\renewcommand{\labelitemiii}{$\bullet$}
\renewcommand{\labelitemiv}{$\bullet$}

%%% Für Programm/Code Listings
\renewcommand{\lstlistingname}{Programm}

%%% Gepunktete Linie im Inhaltsverzeichnis entfernen
\usepackage[titles]{tocloft}
\renewcommand{\cftdot}{}

%%% (Kleinere) Randnotizen
\usepackage{marginnote}
\renewcommand*{\marginfont}{\footnotesize}

%%% Farbenkram
%\usepackage{color}

%%% Gedichte
\usepackage{verse}

%%% PDF Hyperlinks
\usepackage[ngerman,plainpages=false,pdfpagelabels]{hyperref}

%%% biblatex (Quellenverz)
%%% mehr unter http://biblatex.dominik-wassenhoven.de/download/DTK-2_2008-biblatex-Teil1.pdf
%%% benutze biber.exe anstelle von bibtex.exe (miktex)
\usepackage[
	backend=biber,
	bibencoding=utf8,
	style=authortitle-icomp,
	bibstyle=numeric,
	citestyle=authoryear,
	pagetracker=true,
	isbn=false
]{biblatex}
\usepackage[babel,german=quotes]{csquotes}
\setlength{\bibitemsep}{1em}

%%% Das Quellenverz. z.B. mit JabRef erstellt. 
\bibliography{bibliography}
%\addbibresource{bibliography.bib}

%%% Zeilenabstand 1,5
\onehalfspacing

%%% Extra Kommandos
%% \punkte -> ...
\newcommand*{\punkte}{\dots\unkern}

